\documentclass[a4paper,10pt]{book}
\usepackage[utf8]{inputenc}
\usepackage{graphicx}
\usepackage{multicol}
\usepackage[frenchb]{babel}
\usepackage{fancyhdr}
\usepackage{float}


\pagestyle{fancy}
\renewcommand\headrulewidth{0pt}
\renewcommand\footrulewidth{1pt}
\fancyfoot[L]{Projet STCal}
\fancyfoot[R]{IUT Informatique de Belfort}
\lhead{}
\rhead{}
\date{}


\renewcommand{\thesection}{\arabic{part}.\arabic{section}}

\title{\Huge{Projet STCal}\\ {\Large Rapport général}}
\author{Florian Barrois \and Nicolas Devillers \and Valentin Jeanroy \and Mehdi Loisel \and Jean Mercadier \and Ismail Taleb \and Willeme Verdeaux}

\begin{document}
 \begin{titlepage}

    \begin{center}
      ~\\~\\~\\~\\~\\
      \Huge
      Locigiel STCal\\
      \LARGE
      Manuel d'utilisation\\
      ~\\~\\~\\~\\~\\~\\
      \begin{multicols}{3}
	\large
	Florian Barrois\\Mehdi Loisel\\
	\columnbreak 
	Nicolas Devillers\\Jean Mercadier\\Willeme Verdeaux\\
	\columnbreak
	Valentin Jeanroy\\Ismail Taleb
      \end{multicols}
    \end{center}


    \begin{center}
    ~\\~\\~\\~\\~\\~\\~\\~\\~\\~\\~\\~\\
    \includegraphics{../general/iutbm.jpeg}
    ~\\~\\~\\~\\~\\
    \large
    Année 2013-2014
    \end{center}

    \end{titlepage}


\strut
\renewcommand{\contentsname}{Sommaire}
\tableofcontents

\part{Introduction}

\part{Les pré-requis}
\part{La gestion des stages}
    \section{Créer un stage} 
      
      \subsection{Importer les données}
      
	Afin de pouvoir créer des stages, il est nécessaire d'établir au préalable les listes d'informations sur les étudiants et les enseignants.
	Ces listes doivent être importées depuis deux fichiers au format ICS.
	
	\paragraph{}
	\textit{Remarques :
	\renewcommand\labelitemi{\textbullet}
	  \begin{itemize}
	    \item Si vous n'avez aucun fichier possédant cet extension, il est très simple d'en créer en utilisant par exemple un logiciel de tableur.
		  La première ligne devra contenir l'intitulé des champs remplis (ex : Nom étudiant; Prénom étudiant; Lieu de stage;), et dans les autres lignes se trouveront les données.
		  Pour enregistrer ce fichier au format ICS, cliquer sur ``Fichier - Enregistrer sous...'', entrez le nom de votre fichier, puis cliquez sur ``Tous les formats'' et choisissez ``Fichier CSV''.
	    \item En cas de problème d'import, vérifiez que le nombre d'informations sur chaque personne correspond bien au nombre de champs entrés dans la première ligne. 
		  Supprimez éventuellement les lignes vides (contenant seulement des caractères séparateurs).
	  \end{itemize}}

	\paragraph{}
	  Il y a deux manières procéder à l'import de ces fichiers : 
	  ~\\
	  \renewcommand\labelitemi{\textbullet}
	  \begin{itemize}
	    \item Utiliser les boutons présents dans l'onglet Lier.
	    \item Effectuer les imports depuis le menu ``Fichier''.
	  \end{itemize}

	     
	\paragraph{}
	  Prenons la première méthode.
	  Cliquez sur ``Importer étudiants''.
	  Une fenêtre apparaît et vous pouvez alors choisir le fichier CSV à importer.
	  Une fois l'import effectué, la liste des étudiants est présente dans la zone de droite.
	  Répétez ces opérations en cliquant sur ``Importer enseignants''.
	  Vos deux listes sont maintenant affichées et les boutons d'import ne sont plus présents.
	  Il reste toutefois possible d'incorporer d'autres fichiers à l'application depuis le menu ``Fichier''.
	  A la place de ces boutons apparaît alors un troisième intitulé ``Valider'' et qui demeure pour l'instant inaccessible.
	
	\paragraph{}
	\textit{Remarques : 
	  \begin{itemize}
	    \item Il est possible d'importer plusieurs fichiers d'étudiants, ou d'enseignants. La liste de ce type de personne contiendra alors les informations de tous les fichiers importés.
	    \item Vous pouvez sélectionner une personne, étudiant ou enseignant, et connaître toutes les informations qui la concernent depuis le cadre situé dans la zone centrale.
	  \end{itemize}}
	~\\~\\~\\
	
      \subsection{Créer un stage}
 
	\paragraph{}
	  Après avoir importé les fichiers, vous possédez les données requises pour créer un stage.
	  Il reste juste un champ à remplir : celui du diminutif du professeur. 
	  Il est situé juste au-dessus de la zone contenant les informations détaillées de la personne sélectionnée.
	  Ce champ concerne uniquement les responsables de stages, les étudiants ne se verront pas attribuer de diminutif.
	  Sélectionnez un enseignant et octroyez-lui un diminutif.
	  
	
	\paragraph{}
	  Après avoir sélectionné un étudiant et un enseignant auquel a été donné un diminutif, le bouton de validation devient disponible.
	  Actionnez-le et un stage sera créé.
	  L'étudiant disparaît alors de la liste de droite.
	  En revanche, le professeur est toujours présent dans la liste de gauche, et vous pouvez ainsi affecter un autre étudiant à cette personne.
	  
	\paragraph{}
	  \textit{Remarque : Un couple étudiant/professeur est considéré comme ``stage''. Dans ce manuel, toute mention de ``stage'' fera donc référence à ces couples.}
	  
    \section{Consulter la liste des stages}
    \section{Supprimer un stage}
\part{La gestion du calendrier}
    \section{Le paramètrage du calendrier}
    \section{Créer une soutenance}
      \subsection{Créer une soutenance dans le calendrier}
      \subsection{Compléter les informations relatives à la soutenance}
    \section{Consulter les soutenances existantes}
    \section{Supprimer une soutenance}
    \section{Exporter le calendrier}
      \subsection{Exporter au format PDF}
      \subsection{Exporter au format ICS}
\part{Le paramètrage de la base de données}














\end{document}
