\documentclass[a4paper,10pt]{book}
\usepackage[utf8]{inputenc}
\usepackage{graphicx}
\usepackage{multicol}
\usepackage[frenchb]{babel}
\usepackage{fancyhdr}
\usepackage{float}
\usepackage{url}


\pagestyle{fancy}
\renewcommand\headrulewidth{0pt}
\renewcommand\footrulewidth{1pt}
\fancyfoot[L]{\leftmark{\thepart}}
\fancyfoot[R]{IUT Informatique de Belfort}
\lhead{}
\rhead{}
\date{}
\makeatletter
\let\ps@plain=\ps@fancy
\makeatother

\renewcommand{\thesection}{\arabic{part}.\arabic{section}}

\title{\Huge{Projet STCal}\\ {\Large Rapport général}}
\author{Florian Barrois \and Nicolas Devilers \and Valentin Jeanroy \and Mehdi Loisel \and Jean Mercadier \and Ismail Taleb \and Willeme Verdeaux}

\begin{document}
 \begin{titlepage}

    \begin{center}
      ~\\~\\~\\~\\~\\
      \Huge
      Locigiel STCal\\
      \LARGE
      Manuel d'utilisation\\
      ~\\~\\~\\~\\~\\~\\
      \begin{multicols}{3}
	\large
	Florian Barrois\\Mehdi Loisel\\
	\columnbreak 
	Nicolas Devillers\\Jean Mercadier\\Willeme Verdeaux\\
	\columnbreak
	Valentin Jeanroy\\Ismail Taleb
      \end{multicols}
    \end{center}


    \begin{center}
    ~\\~\\~\\~\\~\\~\\~\\~\\~\\~\\~\\~\\
    \includegraphics{../general/iutbm.jpeg}
    ~\\~\\~\\~\\~\\
    \large
    Année 2013-2014
    \end{center}

    \end{titlepage}


\strut
\renewcommand{\contentsname}{Sommaire}
\tableofcontents

\part{Introduction}
  \paragraph{}
    L'application STCal est destinée aux professeurs et aux responsables de stages.
    Elle a pour but de faciliter la gestion des stages en entreprise des étudiants lors de leur cursus universitaire.
    Elle a elle-même été réalisée par notre groupe de sept étudiants en quatrième semestre d'IUT informatique.
    
  \paragraph{}
    Ce manuel a néanmoins été conçu pour un public très large, et ne contient donc pas de références techniques propres à l'informatique.
    Ce document a été écrit à la manière d'un manuel procédural, ce qui signifie que les chapitres ont été disposés dans le sens de la marche à suivre pour utiliser le logiciel.
    La plupart des fonctionnalités seront abordées tout au long de ce plan.
    Les fonctionnalités restantes sont détaillées à la fin du manuel.
  
  \paragraph{}
     Des illustrations seront également placées tout au long du manuel dans le but de faciliter votre compréhension des explications.
    

\part{Les pré-requis}
  \paragraph{}
    Afin de pouvoir lancer l'application et se servir de toutes les fonctionnalités qu'elle propose, il est nécessaire d'avoir certaines archives.
    
  \paragraph{}
    Ces archives sont présentes dans le dossier contenant le fichier exécutable de l'application. 
    Par conséquent, mis à part, un changement de disposition des fichiers de ce dossier, aucun élément supplémentaire n'est à télécharger pour avoir accès au logiciel.

\part{Le paramètrage de la base de données}
  \paragraph{}
    Lors du premier lancement de l'application, le logiciel vous demande de paramétrer la base de données qui sera utilisée pour le stockage de votre travail.
    Entrez les paramètres d'accès à votre base de données, c'est-à-dire votre nom d'utilisateur, votre mot de passe, l'adresse du serveur utilisé et le numéro de port.
    Validez les informations.
    
  \paragraph{}
      \textit{Remarque : Ces données peuvent être changées à tout moment depuis ``Préférences - Base de données''.}
  
  \paragraph{}
    Toutes les fonctions de gestion de la base de données sont disponibles dans l'onglet ``Préférences''.
    En plus des changements des paramètres de connexion à la base de données, cet onglet permet de relancer la connexion à la base.
    Il existe aussi une autre fonctionnalité servant à réinitialiser totalement la base de données.
    Activer cette dernière option efface la base de données, et en crée une nouvelle, vide.
  
\part{La gestion des stages}
    \section{Créer un stage} 
      
      \subsection{Importer les données}
      
	Afin de pouvoir créer des stages, il est nécessaire d'établir au préalable les listes d'informations sur les étudiants et les enseignants.
	Ces listes doivent être importées depuis deux fichiers au format ICS.
	
	\paragraph{}
	\textit{Remarques :
	\renewcommand\labelitemi{\textbullet}
	  \begin{itemize}
	    \item Si vous n'avez aucun fichier possédant cet extension, il est très simple d'en créer en utilisant par exemple un logiciel de tableur.
		  La première ligne devra contenir l'intitulé des champs remplis (ex : Nom étudiant; Prénom étudiant; Lieu de stage;), et dans les autres lignes se trouveront les données.
		  Pour enregistrer ce fichier au format ICS, cliquer sur ``Fichier - Enregistrer sous...'', entrez le nom de votre fichier, puis cliquez sur ``Tous les formats'' et choisissez ``Fichier CSV''.
	    \item En cas de problème d'import, vérifiez que le nombre d'informations sur chaque personne correspond bien au nombre de champs entrés dans la première ligne. 
		  Supprimez éventuellement les lignes vides (contenant seulement des caractères séparateurs). Si le problème persiste, ouvrez le fichier dans un éditeur de texte et vérifiez que les caractères séparants les champs sont des points-virgules. Si ce n'est pas cas, effectuez la correction.
	  \end{itemize}}

	\paragraph{}
	  Il y a deux manières procéder à l'import de ces fichiers : 
	  ~\\
	  \renewcommand\labelitemi{\textbullet}
	  \begin{itemize}
	    \item Utiliser les boutons présents dans l'onglet Lier.
	    \item Effectuer les imports depuis le menu ``Fichier''.
	  \end{itemize}

	     
	\paragraph{}
	  Prenons la première méthode.
	  Dans l'onglet ``Lier'', cliquez sur ``Importer étudiants''.
	  Une fenêtre apparaît et vous pouvez alors choisir le fichier CSV à importer.
	  Une fois l'import effectué, la liste des étudiants est présente dans la zone de droite.
	  Répétez ces opérations en cliquant sur ``Importer enseignants''.
	  Vos deux listes sont maintenant affichées et les boutons d'import ne sont plus présents.
	  Il reste toutefois possible d'incorporer d'autres fichiers à l'application depuis le menu ``Fichier''.
	  A la place de ces boutons apparaît alors un troisième intitulé ``Valider'' et qui demeure pour l'instant inaccessible.
	
	\paragraph{}
	\textit{Remarques : 
	  \begin{itemize}
	    \item Il est possible d'importer plusieurs fichiers d'étudiants, ou d'enseignants. La liste de ce type de personne contiendra alors les informations de tous les fichiers importés.
	    \item Vous pouvez sélectionner une personne, étudiant ou enseignant, et connaître toutes les informations qui la concernent depuis le cadre situé dans la zone centrale.
	  \end{itemize}}
	~\\~\\~\\
	
      \subsection{Créer un stage}
 
	\paragraph{}
	  Après avoir importé les fichiers, vous possédez les données requises pour créer un stage.
	  Il reste juste un champ à remplir : celui du diminutif du professeur. 
	  Il est situé juste au-dessus de la zone contenant les informations détaillées de la personne sélectionnée.
	  Ce champ concerne uniquement les responsables de stages, les étudiants ne se verront pas attribuer de diminutif.
	  Sélectionnez un enseignant et octroyez-lui un diminutif.
	  
	
	\paragraph{}
	  Après avoir sélectionné un étudiant et un enseignant auquel a été donné un diminutif, le bouton de validation devient disponible.
	  Actionnez-le et un stage sera créé.
	  Cette action est également possible dans ``Menu - Lier''.
	  L'étudiant disparaît alors de la liste de droite.
	  En revanche, le professeur est toujours présent dans la liste de gauche, et vous pouvez ainsi affecter un autre étudiant à cette personne.
	  
	\paragraph{}
	  \textit{Remarque : Un couple étudiant/professeur est considéré comme ``stage''. Dans ce manuel, toute mention de ``stage'' fera donc référence à ces couples.}
	  
    \section{Consulter la liste des stages}
      \paragraph{}
	Les stages créés sont consultables dans l'onglet ``Stage''.
	A gauche de la fenêtre se trouve une arborescence dans laquelle sont classés par enseignant les étudiants. 
	Un enseignant est ainsi représenté par un dossier, et les fichiers qu'un dossier contient correspondent aux étudiants dont le stage est suivi par cet enseignant.
	En cliquant sur un étudiant, vous avez, comme dans l'onglet ``Lier'', consulter toutes les informations relatives à cet étudiant. 
	En passant de l'onglet ``Étudiant'' à l'onglet ``Tuteur'' dans la petite zone d'affichage située sur la droite, vous pouvez également avoir accès aux informations de l'enseignant appairé à ce stagiaire.
	
	\paragraph{}
	  \textit{Remarque : Si un professeur n'est associé à aucun étudiant, il n'est pas représenté dans l'arborescence.}
      
    \section{Supprimer un stage}
      \paragraph{}
	Pour supprimer un stage, il suffit d'ouvrir l'onglet ``Stage'', de sélectionner un étudiant, et de cliquer sur le bouton ``Supprimer le stage''.
	Lorsqu'un stage est supprimé, l'étudiant reparaît dans la liste des étudiants de l'onglet ``Lier''.
    
    
\part{La gestion du calendrier}
  \setcounter{section}{0}
    \section{Le paramètrage du calendrier}
      \paragraph{}
	Le paramètrage du calendrier se fait en cinq étapes.
	
      \paragraph{}
	Choisissez d'abord les jours pendant lesquels auront lieu les soutenances sur le calendrier en haut de la page.
	Pour sélectionner plusieurs jours, maintenant la touche Ctrl.
      
      \paragraph{}
	Entrez ensuite l'heure de début et l'heure de fin des journées validées.
	Puis tapez la durée, en minutes, des soutenances.
	Enfin, un créneau peut contenir plusieurs soutenances ayant lieu simultanément dans différentes salles.
	Écrivez donc le nombre maximum de soutenances pouvant se dérouler à une même tranche horaire.
      
      \paragraph{}
	Vous pouvez à présent générer le calendrier en cliquant sur le bouton.
	Une nouvelle interface apparaît et va vous permettre de créer les soutenances des stages.
      
      ~\\~\\
    \section{Créer une soutenance}
    
	\paragraph{}
	  \textit{Remarque : L'interface de la deuxième page de l'onglet ``Calendrier'' peut être perturbée lorsque l'application est réduite. Pour optimiser l'affichage, utilisez le logiciel en mode ``grand écran''.}
    
      \subsection{Créer une soutenance dans le calendrier}
	\paragraph{}
	  L'interface de gestion du calendrier est caractérisée par cinq zones.
	  Dans cette partie, seulement deux de ces zones vont nous intéresser.
	  Pour commencer, la première zone affichant l'arborescence des stages nous permet de choisir le stage dont on veut établir la soutenance.
	  Pour placer la soutenance d'un stage dans le calendrier, sélectionnez l'étudiant dans l'arborescence et, tout en gardant le clic gauche enfoncé, amenez l'étudiant dans le créneau souhaité.
	  L'étudiant disparaît alors de la liste des stages et l'on peut voir son nom et son prénom dans le créneau contenant la soutenance.
	  
	\paragraph{}
	  Néanmoins, pour pouvoir créer une soutenance à partir d'un stage, certaines conditions doivent être respectées.
	  Des créneaux peuvent être inaccessibles, notamment si la salle de soutenance (détaillé ci-après), ou l'un des enseignants assitant à la soutenance, est déjà prévu(e) à cet horaire.
	  C'est aussi le cas lorsque le nombre de soutenance maximal pour un créneau a été atteint.
	  L'accessibilité dépend donc de l'étudiant choisi, et notamment de son tuteur de stage.
	
	\paragraph{}
	  Ainsi, si vous essayez de créer une soutenance à un horaire indisponible, aucun changement ne sera effectué.
	  Cependant vous pouvez repérer très facilement les horaires bloqués.
	  En effet, lorsque vous sélectionnez un étudiant, vous pouvez observer que les cases du calendrier se colorent, soit en rouge lorsque le créneau est indisponible, soit en vert quand le créneau est libre d'occuper cette la soutenance de ce stage.
	  
	\paragraph{}
	  \textit{Remarque : Si vous lancez cette application depuis un terminal informatique, vous bénéficiez des messages d'information de l'application. Par exemple, si vous tentez de créer une soutenance à un horaire incompatible avec le stage choisi, le terminal votre terminal affichera les raisons du refus de création de la soutenance.}
	
	  
	  
	
      \subsection{Compléter les informations relatives à la soutenance}
	\paragraph{}
	  Votre créneau contient maintenant un couple étudiant/professeur, définissant une soutenance.
	  Nous allons maintenant compléter les informations relatives à cette soutenance.
	 
	\paragraph{}
	  En premier lieu, ouvrez le menu ``Préférences - Salles''. 
	  Une petite fenêtre apparaît, révélant la liste des salles implémentées dans le logiciel, qui est pour l'instant vide.
	  Utilisez le champ présent pour créer une nouvelle salle.
	  Indiquez le nom ou le numéro de la salle et ajoutez-la à la liste.
	  
	\paragraph{}  
	  \textit{Remarque : Au passage, remarquez que c'est à cet endroit que s'effectue la configuration des salles, avec la possiblité de les consulter, d'en créer et d'en supprimer.}
	  
	\paragraph{}
	  Une fois votre liste de salle remplie, fermez la fenêtre et portez votre attention sur le coin inférieur droit de l'application.
	  Pour attribuer une salle et un enseignant candide à une soutenance, cliquez d'abord sur le créneau de la soutenance, dans le cadre de droite cliquez ensuite sur la soutenance concernée parmi les soutenances de ce créneau.
	  Vous n'avez maintenant plus qu'à choisir la salle et le professeur candide parmi les choix proposés dans les listes déroulantes.
	  N'oubliez pas de valider votre choix.
	  
	
	~\\~\\
    \section{Consulter les soutenances existantes}
      \paragraph{}
	Pour consulter les soutenances programmées, cliquez simplement sur une plage horaire du calendrier.
	La liste de toutes les soutenances orales prévues à cet horaire s'affiche alors à droite du calendrier.
	Cliquez ensuite sur une de ces soutenances pour afficher les données complémentaires concernant la soutenance.
	Ces données sont indiquées dans le coin inférieur gauche du logiciel, sous l'arborescence des stages.
	Vous pouvez alors consulter le nom du tuteur de stage de l'étudiant, l'enseignant candide présent à la présentation orale, ainsi que le nom, ou numéro, de la salle où cette dernière aura lieu.
      
      ~\\~\\
    \section{Supprimer une soutenance}
      \paragraph{}
	La suppression d'une soutenance se fait grâce au bouton présent sous la liste des soutenances pour un horaire donné.
	Sélectionnez une soutenance dans cet emplacement et cliquez sur ``Supprimer cette soutenance''.
	La soutenance disparaît alors de la liste et du calendrier, tandis que le stage revient dans l'arborescence du tiers gauche de la page.
	Les éléments relatifs à cette ancienne soutenance sont quant à eux effacés.
	Ainsi, si vous recréez la une soutenance identique à celle effacée, les paramètres concernant la salle et l'enseignant candide ne posséderont plus les valeurs précédentes mais seront vides.
	
      \paragraph{}
	\textit{Attention : La suppression d'une soutenance et la suppression d'un stage sont deux choses différentes. Supprimer une soutenance enlève seulement la soutenance du planning. Le stage reste en état tant qu'il n'est pas supprimé depuis l'onglet ``Stage''.}
      
      ~\\~\\
    \section{Exporter le calendrier}
      \subsection{Exporter au format PDF}
      
	\paragraph{}
	  Pour exporter un calendrier, allez dans ``Fichier - Exporter au format ICS ou PDF''.
	  Les données peuvent être triées en fonction de la salle ou du professeur.
	  Elles peuvent aussi être affichées sans tri spécial grâce au champ ``Total'' de la liste déroulante.
	  
	\paragraph{}
	  Si vous souhaitez exporter votre travail dans un fichier PDF, vous devez obligatoirement sélectionner ``Total'', puis ``PDF'.
	  Cliquez ensuite sur ''Exporter``, choisissez l'emplacement de votre nouveau fichier PDF, nommez-le.
	  Une fois ces consignes exécutées, enregistrez et fermez la fenêtre.
	
	\paragraph{}
	  Votre travail est maintenant écrit dans un fichier PDF que vous pouvez consulter directement dans la mémoire de votre ordinateur.

	\paragraph{}
	  \textit{Remarque : Lorsque vous cliquez sur ''Exporter``, la fenêtre de navigation dans les fichiers peut mettre quelques secondes à s'ouvrir.}
	
      \subsection{Exporter au format ICS et ouvrir avec Google}
	  
	\paragraph{}
	  L'export au format ICS se fait de la même manière que l'export au format PDF.
	  Exécutez les directives mentionnées dans la section précédente jusqu'à l'ouverture de la fenêtre d'export.
	  Vous pouvez, pour ce format, choisir le tri que vous souhaitez pour vos données.
	  Sélectionnez ensuite ''ICS`` dans la liste des formats et cliquez sur ''Exporter``.
	  Choisissez votre répertoire de création de votre fichier dans votre ordinateur, enregistrez  et fermez la fenêtre.
	  Votre calendrier est à présent sauvegardé dans le dossier destination que vous avez choisi.
	  
	\paragraph{}
	  Grâce à l'extension ICS de votre fichier, votre calendrier est maintenant compatible avec l'agenda Google.
	  Voyons comment importer votre fichier sur Google.
	  
	  
	\paragraph{}
	  \textit{Remarque : Vous devez posséder un compte Google pour pouvoir effectuer les actions suivantes.}
	
	\paragraph{}
	  Ouvrez un navigateur Internet et recherchez Google Agenda, ou entrez directement l'adresse URL : \url{https://www.google.com/calendar?hl=fr}.
	  ~\\~\\
	  Une fois le calendrier Google ouvert, glissez votre souris sur la flèche à côté de ''Mes agendas``, dans la barre verticale à gauche de l'écran, cliquez et sélectionnez ''Créer un agenda``.
	  Donnez un titre à votre nouveau planning, donnez éventuellement une description et cliquez sur ''Créer l'agenda``.
	  ~\\~\\
	  Vous possédez maintenant un nouvel agenda spécifique à la gestion soutenances orales de stage des étudiants.
	  A gauche, cliquez sur la flèche à côté de ''Autres agendas``, puis sur ''Importer l'agenda``.
	  Dans la nouvelle fenêtre, cliquez sur ''Parcourir`` et sélectionnez votre fichier ICS.
	  Enfin, appuyez sur ''Importer``.
	  
	\paragraph{}
	  Votre calendrier est maintenant sur Google.
	  Vous pouvez alors ajouter, consulter, modifier et supprimer à votre guise des soutenances depuis votre compte Google.
	  
	\newpage
	\paragraph{}
	  Vous connaissez maintenant le logiciel STCal jusqu'au bout des doigts ! 
	  Nous avons passé toutes les fonctionnalités en revue et anticipé tous les soucis potentiels que vous pourriez connaître.
	  
	\paragraph{}
	  Ce manuel est également consultable depuis l'onglet ''? - Aide``.
	  Une brève présentation du projet est également disponible dans ''? - A propos de STCal``.
	

	\paragraph{}
	  Voici la fin de ce manuel.
	  Nous espérons que vous avez pris plaisir à le lire et que notre logiciel vous sera d'une grande aide pour vos projets à venir.









\end{document}
