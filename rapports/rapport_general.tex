\documentclass[a4paper,10pt]{report}
\usepackage[utf8]{inputenc}
\usepackage[frenchb]{babel}


\title{\Huge{Projet Stcal}\\ {\Large Rapport général}}
\author{Florian Barrois}


\begin{document}

\maketitle
\strut
\renewcommand{\contentsname}{Sommaire}
\tableofcontents

\part{Introduction}
  \paragraph{}
    Dans le cadre de notre projet tutoré, nous avons dû réfléchir à un secteur d'activité nécessitant des outils informatiques relativement simples mais essentiels. 
    Ainsi, au vu du besoin éprouvé par nos enseignants de travailler avec des outils plus simples, nous avons choisi de réaliser une application à utilité pédagogique.
    Cette application aura pour but de faciliter la gestion des stages des étudiants de quatrième semestre ainsi que la mise en place d'un calendrier des soutenances orales des étudiants présentant leur stage à leurs professeurs.

  \paragraph{}
    L'application possédera bien sûr des fonctionnalités précises, correspondant aux besoins des enseignants s'occupant des stages, mais ces fonctionnalités seront également conçues de manière à ce que la plupart des établissements d'enseignement supérieur et des organismes gérant des stages puissent s'en servir de façon intuitive. 
    Poursuivant l'objectif de produire un travail complet tout en se mettant dans la peau des utilisateurs, les moyens les plus modernes et adaptés, du point de vue technique, et les plus faciles d'utilisation, du côté de l'usager, seront mis en place pour satisfaire au maximum la demande concernant l'organisation de stages. 

  \paragraph{}
    Nous détaillerons ainsi dans ce rapport toute notre réflexion pour répondre à cette problématique, expliquerons nos choix, et parlerons des avantages et des éventuels inconvénients liés à ces derniers.  
    Des illustrations seront également placées tout au long du rapport dans le but d'apporter une meilleure compréhension et une représentation concrète du travail réalisé.

  \paragraph{}
    Nous allons donc étudier les différents aspects du projet, en commançant par une brève présentation du sujet. 
    Puis nous nous tournerons vers le cahier des charges, qui comprendra les fonctionnalités du produit ainsi que les contraintes à respecter.
    Ensuite, nous observerons plus en détail comment s'est déroulée la réalisation des différentes étapes.
    Enfin nous aborderons les difficultés survenues et concluerons sur le résultat obtenu et sur le projet en lui-même.



\part{Présentation du sujet}
  \paragraph{}
    Depuis de nombreuses années, les professeurs de notre établissement, et sûrement bien d'autres, s'occupent des stages en s'adaptant aux outils existants.
    En effet, à ce jour, la méthode employée pour réaliser cette tâche se compose de la gestion d'une part des informations, et d'autre part du planning des soutenances, en utilisant respectivement des logiciels tels que Microsoft Excel, et les fonctionnalités de calendrier de Google.
    La mission dont nous nous sommes acquittés consiste donc à aider les responsables de stage, non pas par la recherche d'outils plus appropriés, mais par la création d'une application qui sera spécifiquement prévue à cet usage et qui demeurera parfaitement adaptée aux besoins des usagers, notamment du personnel enseignant.

  \paragraph{}
    Ainsi la difficulté de ce défi est conséquente, puisque nous ne connaissons aucun outil similaire existant pouvant servir de référence ou permettant d'étayer notre réflexion.
    Notre groupe de sept étudiants sera donc chargé de trouver la bonne démarche à suivre en mettant à profit toutes les connaissances, qu'elles aient été acquises en cours ou à la suite de recherches liées au projet.
    Les connaissances utilisées relèveront de la mise en place d'une interface graphique, de la gestion de classes et de bases de données, en plus de toutes les notions apprises liées au langage utilisé, à savoir le Java.
  
  \paragraph{}
    Le résultat ne consistera en aucun cas en une plate-forme de consultation des stages.
    L'application développée sera mise à disposition exclusive des enseignants. 
    Les étudiants concernés n'auront quant à eux aucun accès à l'application, que ce soit pour ajouter, supprimer, consulter ou éditer des données.

%Le sujet est divisé en plusieurs étapes. 
%Tout d'abord, il est nécessaire de d'établir une liste des stagiaires et une liste des professeurs et ainsi attribuer à chaque étudiant un enseignant dit responsable de stage.


    
    
    
\part{Cahier des charges}
  
  \chapter{Fonctionnalités}

    L'application Stcal devra simplifier la gestion des stages des étudiants par les professeurs grâce à plusieurs fonctionnalités, réparties ci-dessous en quatre étapes  :

    \begin{enumerate}
      \item La création de stages.
      \item La création et la gestion d’un emploi du temps.
      \item La création d'une base de données. 
      \item L'existence d'un système de sauvegarde.
    \end{enumerate}
  
  
  
  
  % Association enseignant/étudiant
  %     Contrainte limitant l'association d'un étudiant à un seul enseignant.
  % Gestion du planning des soutenances de stage
  %   Affichage de l'horaire choisi lors de la sélection du couple enseignant/étudiant ?
  %   Affichage du couple enseignant/étudiant lors de la sélection de l'horaire choisi ?
  % Drag & drop des étudiants et enseignants sur l'horaire choisi ?

  \chapter{Développement et contraintes} % A RÉORGANISER

    Le développement s'effectuera en quatre étapes, qui contiendront différentes tâches. Ces étapes et tâches seront ordonnées tel que le côté fonctionnel de l'application soit prioritaire :

    \section{Première étape}

      \subsection{Développement}
	\paragraph{}
	L'application devra avant tout considérer des étudiants et des professeurs.
	\newline
	Le produit permettra ainsi l'import de fichiers contenant les données. 
	Ces dernières seront par la suite utilisées lors de la création de couples professeur/étudiant. 
  
      \subsection{Contraintes}
	\paragraph{}
	Détaillons maintenant les différentes contraintes à respecter concernant cette étape. 
	\paragraph{}
	A partir d'une liste de professeurs et d'une liste d'étudiants existantes, un bouton devra permettre d'associer un étudiant à un professeur.
	N'ayant qu'un tuteur de stage, chaque étudiant sera relié à un et un seul enseignant.
	En revanche, un enseignant peut posséder la fonction de responsable de stages pour plusieurs élèves. 
	Chaque enseignant pourra donc être relié à aucun, un ou plusieurs étudiants.	
      
	
    \section{Deuxième étape}

      \subsection{Développement}
	~\\
	\paragraph{}
	L’application comportera une fonction de création d’emploi du temps vide et une seconde permettant le remplissage de celui-ci avec les soutenances de chacun des étudiants.
	Les plages horaires réservées contiendront les informations suivantes :
	\newline
	- Le nom de l’enseignant tuteur
	\newline
	- Le nom de l’enseignant candide 
	\newline
	- Le nom de l’étudiant
	\newline
	- Le numéro de la salle d’entretien

	\paragraph{}
	Le logiciel possèdera également une option d’export du calendrier généré au format ICS compatible avec Google et iCal.
	
      \subsection{Contraintes}
	\paragraph{}

	Le produit devra être capable de résoudre les cas suivants en affichant un message d’erreur et en ignorant la dernière action effectuée :
	- en cas de création d'une soutenance à un horaire incompatible pour l'enseignant tuteur ;
	\newline
	- en cas de création d'une soutenance à un horaire incompatible pour l'enseignant candide ;
	\newline
	- en cas de création d'une soutenance ayant lieu dans une salle déjà réquisitionnée à cet horaire ;
	\newline
	- si toutes les salles d’entretien sont occupées à un horaire donné, cette plage horaire devra être inaccessible pour placer d’autres soutenances.

      

    \section{Troisième étape}
      \subsection{Développement}
	\paragraph{}
	Le produit possèdera un outil de création de base de données dans laquelle les stages pourront être stockés. Les informations enregistrées seront identiques à celles affichées sur les horaires de soutenances. 
	La base de données créée pourra ensuite être reliée à un client léger, comme un site web dynanmique, et sera consultable par les étudiants.

      \subsection{Contraintes}
	\paragraph{}
	La contrainte principale de cette étape consiste à éviter la présence de doublons dans la base données. 
	Le logiciel devra donc afficher un message d'erreur si les données telles que le nom et le prénom d'une personne, étudiant ou enseignant, entrées dans l'application sont déjà présentes dans la base de données. 
      
    \section{Quatrième étape}
      \subsection{Développement}
	\paragraph{}
      L’application contiendra un système de sauvegarde : lors de l’ouverture de l’application par l’utilisateur, le système effectuera une restauration de session qui permettra l’affichage de la dernière page affichée avant la fermeture de l’application lors la session de travail précédente.

      \subsection{Contraintes}
	\paragraph{}
      Lors de la première utilisation du logiciel, l'utilisateur devra entrer les paramètres correspondant à la base de données qui stockera les informations, à savoir le nom du serveur, le port utilisé, le nom d'utilisateur et le mot de passe.
      Une fois les champs renseignés, ces paramètres seront également mémorisés et la restauration de session devra passer par un accès cette base de données pour afficher les informations précédemment enregistrées.
      Le programme sera donc contraint de rechercher l'existence d'un fichier de configuration dans l'ordinateur afin de savoir si l'application est utilisée ou non pour la première fois.
      Si c'est le cas, une fenêtre contenant les différents paramètres à renseigner mentionnés précédemment apparaît, et les valeurs entrées sont mémorisées dans un nouveau fichier de configuration.
      ~\\
      ~\\
      ~\\
	\paragraph{}
	  Ces étapes consistuent ainsi le fil conducteur du projet. 
	  Il faut également noter l'existence d'une autre étape effectuée en parallèle : la réalisation de l'interface graphique.
	  En effet, celle-ci offre la possibilité d'entrevoir le rendu final qui sera livré au client et facilite également la visibilité quant à l'évolution du projet.
	  Elle permet aussi d'incorporer directement dans le logiciel les fonctions programmées en interne au fur et à mesure que le projet progresse et tient ainsi le rôle de de plate-forme de tests.   
	
\part{Réalisation}
  \setcounter{chapter}{0}
  \chapter{Chap 1}	
    \section{Une section}
  \chapter{Chap 2}
    \section{Une section}
\part{Difficultés rencontrées}
  \chapter{Difficultés techniques}
  \chapter{Répartition des tâches}
  \chapter{Organisation}
Lorem ipsum dolor sit amet, consectetur adipisicing elit, sed do eiusmod tempor incididunt ut labore et dolore magna aliqua. Ut enim ad minim veniam, quis nostrud exercitation ullamco laboris nisi ut aliquip ex ea commodo consequat. Duis aute irure dolor in reprehenderit in voluptate velit esse cillum dolore eu fugiat nulla pariatur. Excepteur sint occaecat cupidatat non proident, sunt in culpa qui officia deserunt mollit anim id est laborum.

\part{Conclusion}
\end{document}          
