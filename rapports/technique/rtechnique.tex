\documentclass[a4paper,10pt]{report}

\usepackage[utf8]{inputenc}
\usepackage[frenchb]{babel}
\usepackage[T1]{fontenc}
\usepackage[top=2cm, bottom=2cm, left=2cm, right=2cm]{geometry}
\usepackage{listings}


\title{Rapport Technique}

\author{Florian Barrois \and Nicolas Devillers \and Valentin Jeanroy \and Mehdi Loisel \and Jean Mercadier \and Ismail Taleb \and Willeme Verdeaux}

\lstset{
language=Java,
basicstyle=\footnotesize,
numbers=left,
numberstyle=\normalsize,
numbersep=5pt,
}

\begin{document}

\thispagestyle{headings}

\maketitle

\tableofcontents

\chapter*{Introduction}

\chapter{Installation}

\section{Lancer l'application}

\subsection{Prérequis}

	Pour lancer l'application, il est obligatoire d'avoir à diposition un server MySQL. Ainsi au lancement le programme chargera ses information dans la base de donné. Mais si la base de donné n'existe pas celci sera cree avec le nom de stcal.


	Cette base de donnée permet ainsi à l'application de sauvegarder et recharger ses donné au lancement et à la fin de l'execution du programe.

\paragraph*{note:}
	Il est conseillé de creer un utilisateur specifique au programme qui aura tout les privilleges sur la base de donné stcal.

\chapter{Enviroment de devellopment}


\chapter{Modelisation}

\chapter*{Conclusion}

\end{document}